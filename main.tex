%! TeX Program = lualatex

\documentclass[letterpaper,11pt]{article}

\usepackage{latexsym}
\usepackage[empty]{fullpage}
\usepackage{titlesec}
\usepackage{marvosym}
\usepackage[usenames,dvipsnames]{xcolor}
\usepackage{verbatim}
\usepackage{enumitem}
\usepackage[hidelinks]{hyperref}
\usepackage[margin = 0.5in]{geometry}
\usepackage{fancyhdr}
\usepackage[english]{babel}
\usepackage{tabularx}
\usepackage{wrapfig}

%--------------UNDERLINING---------------
% taken from https://alexwlchan.net/2017/10/latex-underlines/

\usepackage{contour}
\usepackage{ulem}

\renewcommand{\ULdepth}{3.8pt}
\contourlength{0.8pt}

\newcommand{\myuline}[1]{%
  \uline{\phantom{#1}}%
  \llap{\contour{white}{#1}}%
}

%----------FONT OPTIONS----------
\usepackage{fontspec}
\usepackage{fontawesome5}

% I could have just used the charter package here for the font I wanted...
% but I'd still need to change compiler to XeLaTeX to include the title font
\usepackage{charter}

\setmainfont{XCharter}
\let\emph\relax
\DeclareTextFontCommand{\emph}{\itshape}

\pagestyle{fancy}
\fancyhf{} % clear all header and footer fields
\fancyfoot{}
\renewcommand{\headrulewidth}{0pt}
\renewcommand{\footrulewidth}{0pt}
\setlength{\footskip}{5pt}

% Adjust margins
\iffalse
%\iftrue
\addtolength{\oddsidemargin}{0.3in}
\addtolength{\marginparsep}{0.3in}
\addtolength{\evensidemargin}{0.3in}
\addtolength{\textwidth}{-0.5in}
\addtolength{\topmargin}{0.3in}
\addtolength{\textheight}{0.3in}
%\fi
\fi

\urlstyle{same}

\raggedright
\setlength{\tabcolsep}{0in}

% Sections formatting
\titleformat{\section}{
    {\color{gray}\titlerule} \vspace{-5pt}
    \vspace{8pt}\scshape\raggedright\normalfont
}{}{0pt}{}

%-------------------------
% Custom commands
\newcommand{\resumeItem}[1]{
  \item\small{
    {#1 \vspace{-2pt}}
  }
}

\newcommand{\resumeSubheading}[4]{
  \vspace{-2pt}\item
    \begin{tabular*}{0.97\textwidth}[t]{l@{\extracolsep{\fill}}r}
      \textbf{#1} & #2 \\
      \textit{\small#3} & \textit{\small #4} \\
    \end{tabular*}\vspace{-7pt}
}

\newcommand{\resumeSubSubheading}[2]{
    \item
    \begin{tabular*}{0.97\textwidth}{l@{\extracolsep{\fill}}r}
      \textit{\small#1} & \textit{\small #2} \\
    \end{tabular*}\vspace{-7pt}
}

\newcommand{\resumeProjectHeading}[2]{
    \item
    \begin{tabular*}{0.97\textwidth}{l@{\extracolsep{\fill}}r}
      \small#1 & #2 \\
    \end{tabular*}\vspace{-7pt}
}

\newcommand{\resumeSubItem}[1]{\resumeItem{#1}\vspace{-4pt}}

\renewcommand\labelitemii{$\vcenter{\hbox{\tiny$\bullet$}}$}

\newcommand{\resumeSubheadingListStart}{\begin{itemize}[leftmargin=0.15in, label={}]}
\newcommand{\resumeSubheadingListEnd}{\end{itemize}}
\newcommand{\resumeItemListStart}{\begin{itemize}}
\newcommand{\resumeItemListEnd}{\end{itemize}\vspace{-5pt}}


% PII
\newcommand{\anonymize}{0}

\ifodd\anonymize
    \newcommand{\myemail}{email@gmail.com}
    \newcommand{\myname}{First Last}
    \newcommand{\mygh}{github.com/first-mi-last}
    \newcommand{\mylinkedin}{linkedin.com/in/fimi-last}
    \newcommand{\myphone}{314-159-2653}
\else
    \newcommand{\myemail}{jonathanzhang929@gmail.com}
    \newcommand{\myname}{Jonathan Zhang}
    \newcommand{\mygh}{github.com/jonathan-d-zhang}
    \newcommand{\mylinkedin}{linkedin.com/in/jd-zhang}
    \newcommand{\myphone}{484-986-8012}
\fi

\begin{document}

{\Huge \myname}
\\
\vspace{2pt}

\small{
    \faEnvelope
    \thinspace \thinspace
    \myemail
    \thinspace \thinspace $|$
    \faGithub
    \thinspace \thinspace
    \mygh
    \thinspace \thinspace $|$
    \faLinkedin
    \thinspace \thinspace
    \mylinkedin
    \thinspace \thinspace $|$
    \faPhone
    \thinspace \thinspace
    \myphone
}

\vspace{-11pt}

\section{Education}
  \resumeSubheadingListStart
    \resumeSubheading{Temple University}{Aug. 2022 -- Present}{B.S. Computer Science}{Expected Graduation May 2026}
    \resumeItemListStart
        \resumeItem{3.9 GPA, Dean's List Fall 2022 and Spring 2023}
    %     \resumeItem{Relevant Coursework:}
    %     \resumeItemListStart
    %     \resumeItemListEnd
    \resumeItemListEnd
  \resumeSubheadingListEnd

\section{Technologies}
 \begin{itemize}[leftmargin=0.15in, label={}]
    \small{\item{
    \textbf{Programming Languages}{:
        Python,
        Rust,
        C\#,
        Java,
        C,
        SQL,
        HTML,
        CSS.
    } \\
    \textbf{Libraries}{:
        FastAPI,
        ASP.NET,
        Entity Framework,
        pytest,
        Juniper,
        NumPy,
        pandas.
    } \\
    \textbf{Tools}{:
        git,
        GitHub Actions,
        Docker,
        RabbitMQ,
        gdb,
        Grafana,
        Prometheus,
        Redis,
        Sentry.
    }
    %  \textbf{Skills}{: TDD}
    }}
 \end{itemize}


\section{Work Experience}
    \resumeSubheadingListStart
        \resumeSubheading{Epicor}{May 2023 -- Present}{Software Engineer Intern}{}
        \resumeItemListStart
            % ideally, we get some sort of KPI here. e.g., a survey of users
            % afterwards, or google performance analytics or something.
            \resumeItem{Migrate internal VB.net web apps to Blazor Webassembly, improving responsiveness and user experience.}
            \resumeItem{Streamlined the on-boarding process, creating a single source of up-to-date information for new team members.}
            \resumeItem{Optimized several inefficient endpoints by applying database indexes, caching, and compression to reduce latency by an average of 78\%.}
            \resumeItem{Collaborated with the DevOps team to create a CI/CD pipeline that eliminated an estimated 24 man-hours per quarter.}
        \resumeItemListEnd

        \resumeSubheading{Chipotle Mexican Grill}{June 2022 -- May 2023}{Crew Member}{}
        \resumeItemListStart
            \resumeItem{Wash and cut food in order to prepare the restaurant for opening.}
        \resumeItemListEnd
    \resumeSubheadingListEnd


% :clap: demonstrated :clap: skills :clap:
% Focus on impact
\section{Projects}
    \resumeSubheadingListStart
        \resumeProjectHeading{\textbf{Dragonfly (VipyrSec)} $|$ \emph{Python, SQLAlchemy, FastAPI, Rust, Auth0, Sentry}}{}
        \resumeItemListStart
            \resumeItem{Dragonfly automatically scans new releases on the Python Package Index for malicious files.}
            \resumeItem{Among the fastest and most consistent reporters for malicious packages according to PyPI Director of Infrastructure.}
            \resumeItem{Work directly with the Python Software Foundation to improve malware reporting infrastructure.}
            \resumeItem{Collaborate with a team of 15+ developers, cybersecurity experts, and DevOps engineers.}
            % Get availability measurements
            \resumeItem{Designed and implemented a distributed client-server architecture focused on scalability and fault tolerance.}
            \resumeItem{Successfully reported over 575 malicious packages and counting.}
        \resumeItemListEnd

        \resumeProjectHeading{\textbf{Blog} $|$ \emph{Rust, Docker, Github Actions, Grafana, nginx, HTML and CSS}}{}
        \resumeItemListStart
            % FIXME Boring "used x" bullets
            \resumeItem{Continuously deployed new articles to a VPS using Github Actions.}
            \resumeItem{Monitored metrics and logs using Grafana and Prometheus.}
            \resumeItem{Created a reverse proxy with nginx to host the blog and Grafana on the same domain.}
            \resumeItem{Used HTML, CSS, and Handlebars.js templates for the frontend.}
        \resumeItemListEnd

        \resumeProjectHeading{\textbf{Cryptograms} $|$ \emph{Rust, GraphQL, Docker, PostgreSQL}}{}
        \resumeItemListStart
            % FIXME Boring "used x" bullets
            \resumeItem{Cryptograms is a GraphQL API that provides encrypted puzzles to solve.}
            \resumeItem{Implemented a custom test harness for setup/teardown functionality that was missing in the built-in harness.}
            \resumeItem{Maintained test coverage greater than 90\% using Github Actions, with both unit and integration tests.}
            \resumeItem{Tracked active cryptograms with PostgreSQL.}
        \resumeItemListEnd

        \resumeProjectHeading{\textbf{LiveCC (OwlHacks 2023)} $|$ \emph{Java, Python, FastAPI, JavaFX, Redis, Apache HttpClient, Docker}}{}
        \resumeItemListStart
            \resumeItem{LiveCC provides real time closed captioning for use in classrooms and lecture halls.}
            \resumeItem{Led a team of 3 others to deliver the finished app on time with all planned features.}
            \resumeItem{Created a JavaFX app that uploads audio using Apache HttpClient to a REST API for transcription.}
            \resumeItem{Persisted transcripts using Redis streams, which allowed for efficient appending and distribution.}
        \resumeItemListEnd

%        \resumeProjectHeading{\textbf{HTTP Server} $|$ \emph{C, sockets}}{}
%        \resumeItemListStart
%            \resumeItem{Implements HTTP 1.1 according to RFC 2616.}
%            \resumeItem{Created a custom threadpool in order to handle multiple connections in parallel.}
%            \resumeItem{\dots}
%            \resumeItem{\dots}
%        \resumeItemListEnd
    \resumeSubheadingListEnd
\end{document}
