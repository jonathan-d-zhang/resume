%! TeX Program = lualatex

% Based off of: https://github.com/sb2nov/resume and Jake Gutierrez's version
% of it at https://www.overleaf.com/latex/templates/jakes-resume/syzfjbzwjncs
% inspired by https://practicaltypography.com/resumes.html

% Copyright © 2020 Jia Yang

% Permission is hereby granted, free of charge, to any person obtaining a copy of
% this software and associated documentation files (the “Software”), to deal in
% the Software without restriction, including without limitation the rights to
% use, copy, modify, merge, publish, distribute, sublicense, and/or sell copies
% of the Software, and to permit persons to whom the Software is furnished to do
% so, subject to the following conditions:

% The above copyright notice and this permission notice shall be included in all
% copies or substantial portions of the Software.

% THE SOFTWARE IS PROVIDED “AS IS”, WITHOUT WARRANTY OF ANY KIND, EXPRESS OR
% IMPLIED, INCLUDING BUT NOT LIMITED TO THE WARRANTIES OF MERCHANTABILITY,
% FITNESS FOR A PARTICULAR PURPOSE AND NONINFRINGEMENT. IN NO EVENT SHALL THE
% AUTHORS OR COPYRIGHT HOLDERS BE LIABLE FOR ANY CLAIM, DAMAGES OR OTHER
% LIABILITY, WHETHER IN AN ACTION OF CONTRACT, TORT OR OTHERWISE, ARISING FROM,
% OUT OF OR IN CONNECTION WITH THE SOFTWARE OR THE USE OR OTHER DEALINGS IN THE
% SOFTWARE.

\documentclass[letterpaper,11pt]{article}

\usepackage{cmap} % Helps with ATS, allegedly
\usepackage{latexsym}
\usepackage[empty]{fullpage}
\usepackage{titlesec}
\usepackage[english]{babel}
\usepackage[usenames,dvipsnames]{xcolor}
\usepackage{verbatim}
\usepackage{enumitem}
\usepackage[margin = 0.5in]{geometry}
\usepackage{calc}
\usepackage{fancyhdr}
\usepackage{tabularx}
\usepackage{src/resume}
%\usepackage{luacode}

%----------FONT OPTIONS----------
\usepackage{fontspec}
\usepackage{fontawesome5}

% I could have just used the charter package here for the font I wanted...
% but I'd still need to change compiler to XeLaTeX to include the title font
%\usepackage{charter}

\setmainfont{XCharter}
\let\emph\relax
\DeclareTextFontCommand{\emph}{\itshape}

\pagestyle{fancy}
\fancyhf{} % clear all header and footer fields
\fancyfoot{}
\renewcommand{\headrulewidth}{0pt}
\renewcommand{\footrulewidth}{0pt}
\setlength{\footskip}{5pt}

\raggedright
\setlength{\tabcolsep}{0in}

% Sections formatting
\titleformat{\section}
{\scshape\raggedright\normalfont} % format
{} % label
{0pt} % sep
{{\color{gray}\titlerule}\\} % before-code

\titlespacing{\section}
{0pt} % left
{0pt} % before
{6pt} % after

\begin{document}

{\Huge \myname}
\\
\vspace{2pt}

\small{
    \faEnvelope
    \thinspace \thinspace
    \myemail
    \thinspace \thinspace $|$
    \faGithub
    \thinspace \thinspace
    \mygh
    \thinspace \thinspace $|$
    \faLinkedin
    \thinspace \thinspace
    \mylinkedin
    \thinspace \thinspace $|$
    \faPhone
    \thinspace \thinspace
    \myphone
    \thinspace \thinspace $|$
    \faIcon{map-marker-alt}
    \thinspace \thinspace
    \mylocation
}

\vspace{-7pt}

\section{Education}
\begin{entries}
    \item\begin{experience}{Temple University}{Aug. 2022 -- Present}{B.S. Computer Science}{Expected Graduation May 2025}
        \item 3.95 GPA
        \item Dean's List Fall 2022, Spring 2023, Fall 2023, Spring 2024
    \end{experience}
\end{entries}

\section{Technologies}
\begin{description}[nosep, labelindent=0.15in]
    \item[Programming Languages:] \myprogramminglanguages.
    \item[Libraries:] \mylibraries.
    \item[Tools:] \mytools.
\end{description}

\section{Work Experience}
\begin{entries}
    \item\begin{experience}{Comcast}{May 2024 -- Present}{Software Engineer Intern}{}
        \item Implemented a team-wide documentation style guide focused on
            readability and accessibility.
    \end{experience}

    \item\begin{experience}{Epicor}{May 2023 -- May 2024}{Database Developer Intern}{}
        \item Migrated legacy applications to Blazor WebAssembly (WASM), requiring
            significant research into application specific dependencies and
            creation of clear documentation.
        \item Over 90\% of users prefer the new WASM apps compared to the legacy
            apps.
        \item Streamlined the onboarding process, creating a centralized,
            up-to-date information resource for new team members, successfully
            onboarding a new developer to the project.
        \item Optimized several inefficient endpoints by applying database indexes,
            caching, and compression to reduce P95 latency by 68\%.
        \item Collaborated with the DevOps team to create a CI/CD pipeline that
            replaced approximately 24 man-hours per quarter.
    \end{experience}
\end{entries}


% 👏 demonstrated 👏 skills 👏
% Focus on impact
\section{Projects}
\begin{entries}
    \item\begin{project}{\textbf{Vipyr Security} $|$ \emph{Python, Rust, FastAPI, Kubernetes, PostgreSQL}}{}
        \item VipyrSec automatically scans new releases on the Python
            Package Index for malicious signatures.
        \item Designed and implemented a distributed microservice architecture
            focused on scalability and fault tolerance.
        \item Scans an average of 5300 lines of code per second and up to 10000
            per second at peak volume.
%        \item Manage a PostgreSQL database
        \item Manage and monitor weekly production deployments to Digital Ocean
            Kubernetes using Helm and Grafana dashboards.
        \item Collectively reported over 1000 malicious packages as one of
            the fastest and most consistent reporters according to the
            Python Software Foundation's Director of Infrastructure.
        % Get availability measurements
    \end{project}

    \item\begin{project}{\textbf{Mastodata} $|$ \emph{Kubernetes, ZeroMQ, Grafana, Python, Rust}}{}
        % FIXME: Boring "used x" bullets
        \item Mastodata aggregates and analyzes data from Mastodon statuses.
        \item Distributed tasks via ZeroMQ to worker nodes written in Python.
        \item Used Helm to manage releases to Digital Ocean Managed Kubernetes.
        \item Publishes data to Questdb for viewing with Grafana.
    \end{project}

    \item\begin{project}{\textbf{LiveCC (OwlHacks 2023)} $|$ \emph{Java, Python, FastAPI, JavaFX, Redis, Apache HttpClient, Docker}}{}
        \item LiveCC provides real time closed captioning for use in
            classrooms and lecture halls.
        \item Led 3 developers to deliver the finished app on time with all
            planned features.
        \item Created a JavaFX app that uploads audio using Apache
            HttpClient to a REST API for transcription.
        \item Persisted transcripts using Redis streams, which allowed for
            efficient appending and distribution.
    \end{project}
\end{entries}
\end{document}
