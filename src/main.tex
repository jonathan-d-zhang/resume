%! TeX Program = lualatex

% Based off of: https://github.com/sb2nov/resume and Jake Gutierrez's version
% of it at https://www.overleaf.com/latex/templates/jakes-resume/syzfjbzwjncs
% inspired by https://practicaltypography.com/resumes.html

% Copyright © 2020 Jia Yang

% Permission is hereby granted, free of charge, to any person obtaining a copy of
% this software and associated documentation files (the “Software”), to deal in
% the Software without restriction, including without limitation the rights to
% use, copy, modify, merge, publish, distribute, sublicense, and/or sell copies
% of the Software, and to permit persons to whom the Software is furnished to do
% so, subject to the following conditions:

% The above copyright notice and this permission notice shall be included in all
% copies or substantial portions of the Software.

% THE SOFTWARE IS PROVIDED “AS IS”, WITHOUT WARRANTY OF ANY KIND, EXPRESS OR
% IMPLIED, INCLUDING BUT NOT LIMITED TO THE WARRANTIES OF MERCHANTABILITY,
% FITNESS FOR A PARTICULAR PURPOSE AND NONINFRINGEMENT. IN NO EVENT SHALL THE
% AUTHORS OR COPYRIGHT HOLDERS BE LIABLE FOR ANY CLAIM, DAMAGES OR OTHER
% LIABILITY, WHETHER IN AN ACTION OF CONTRACT, TORT OR OTHERWISE, ARISING FROM,
% OUT OF OR IN CONNECTION WITH THE SOFTWARE OR THE USE OR OTHER DEALINGS IN THE
% SOFTWARE.

\documentclass[letterpaper,11pt]{article}

\usepackage{latexsym}
\usepackage[empty]{fullpage}
\usepackage{titlesec}
\usepackage[english]{babel}
\usepackage[usenames,dvipsnames]{xcolor}
\usepackage{verbatim}
\usepackage{enumitem}
\usepackage[margin = 0.5in]{geometry}
\usepackage{calc}
\usepackage{fancyhdr}
\usepackage{tabularx}
\usepackage{src/resume}

%----------FONT OPTIONS----------
\usepackage{fontspec}
\usepackage{fontawesome5}

% I could have just used the charter package here for the font I wanted...
% but I'd still need to change compiler to XeLaTeX to include the title font
%\usepackage{charter}

\setmainfont{XCharter}
\let\emph\relax
\DeclareTextFontCommand{\emph}{\itshape}

\pagestyle{fancy}
\fancyhf{} % clear all header and footer fields
\fancyfoot{}
\renewcommand{\headrulewidth}{0pt}
\renewcommand{\footrulewidth}{0pt}
\setlength{\footskip}{5pt}

\raggedright
\setlength{\tabcolsep}{0in}

% Sections formatting
\titleformat{\section}
{\scshape\raggedright\normalfont} % format
{} % label
{0pt} % sep
{{\color{gray}\titlerule}\\} % before-code

\titlespacing{\section}
{0pt} % left
{0pt} % before
{6pt} % after

% PII
\ifdefined\isanonymous
    \newcommand{\myemail}{firstlastnumbers@gmail.com}
    \newcommand{\myname}{First Lastg}
    \newcommand{\mygh}{first-mi-last}
    \newcommand{\mylinkedin}{fimi-last}
    \newcommand{\myphone}{314-159-2653}
    \newcommand{\mylocation}{United States}
\else
    \newcommand{\myemail}{jonathanzhang929@gmail.com}
    \newcommand{\myname}{Jonathan Zhang}
    \newcommand{\mygh}{jonathan-d-zhang}
    \newcommand{\mylinkedin}{jd-zhang}
    \newcommand{\myphone}{484-986-8012}
    \newcommand{\mylocation}{Philadelphia, PA}
\fi

\begin{document}

{\Huge \myname}
\\
\vspace{2pt}

\small{
    \faEnvelope
    \thinspace \thinspace
    \myemail
    \thinspace \thinspace $|$
    \faGithub
    \thinspace \thinspace
    \mygh
    \thinspace \thinspace $|$
    \faLinkedin
    \thinspace \thinspace
    \mylinkedin
    \thinspace \thinspace $|$
    \faPhone
    \thinspace \thinspace
    \myphone
    \thinspace \thinspace $|$
    \faIcon{map-marker-alt}
    \thinspace \thinspace
    \mylocation
}

\vspace{-8pt}

\section{Education}
\begin{entries}
    \item\begin{experience}{Temple University}{Aug. 2022 -- Present}{B.S. Computer Science}{Expected Graduation May 2026}
        \item 3.93 GPA, Dean's List Fall 2022, Spring 2023, and Fall 2023
    \end{experience}
\end{entries}

\section{Technologies}
\begin{description}[nosep, labelindent=0.15in]
    \item[Programming Languages:]
        Python,
        Rust,
        C\#,
        Java,
        C,
        SQL,
        HTML,
        CSS.
    \item[Libraries:]
        FastAPI,
        ASP.NET,
        Entity Framework,
        pytest,
        Juniper,
        NumPy,
        pandas.
    \item[Tools:]
        Docker,
        Kubernetes,
        ZeroMQ,
        Grafana,
        Prometheus,
        Redis,
        Sentry,
        gdb,
        git,
        GitHub Actions.
\end{description}

\section{Work Experience}
\begin{entries}
    \item\begin{experience}{Epicor}{May 2023 -- Present}{Software Engineer Intern}{}
        \item Migrate legacy applications to Blazor WebAssembly (WASM), requiring
            significant research into application specific dependencies and
            creation of clear documentation.
        \item Over 90\% of users prefer the new WASM apps compared to the legacy
            apps.
        \item Streamlined the on-boarding process, creating a single source of
            up-to-date information for new team members.
        \item Optimized several inefficient endpoints by applying database indexes,
            caching, and compression to reduce P95 latency by 68\%.
        \item Collaborated with the DevOps team to create a CI/CD pipeline that
            replaced approximately 24 man-hours per quarter.
    \end{experience}
\end{entries}


% 👏 demonstrated 👏 skills 👏
% Focus on impact
\section{Projects}
\begin{entries}
    \item\begin{project}{\textbf{Vipyr Security} $|$ \emph{Python, Rust, RabbitMQ, FastAPI, Sentry}}{}
        \item VipyrSec automatically scans new releases on the Python
            Package Index (PyPI) for malicious signatures.
        \item Designed and implemented a distributed client-server
            architecture focused on scalability and fault tolerance.
        \item Used RabbitMQ to communicate between components in a fan-out/fan-in pattern.
        \item Successfully reported over 650 malicious packages as one of
            the fastest and most consistent reporters according to the
            Python Software Foundation's Director of Infrastructure.
        % Get availability measurements
    \end{project}

    \item\begin{project}{\textbf{Mastodata} | \emph{Kubernetes, ZeroMQ, Grafana, Python, Rust}}{}
        % FIXME: Boring "used x" bullets
        \item Mastodata aggregates and analyzes data from Mastodon statuses.
        \item Distributed tasks via ZeroMQ to worker nodes written in Python.
        \item Used Helm to manage releases to Digital Ocean Managed Kubernetes.
        \item Publishes data to Questdb for viewing with Grafana.
    \end{project}

    \item\begin{project}{\textbf{LiveCC (OwlHacks 2023)} $|$ \emph{Java, Python, FastAPI, JavaFX, Redis, Apache HttpClient, Docker}}{}
        \item LiveCC provides real time closed captioning for use in
            classrooms and lecture halls.
        \item Led 3 developers to deliver the finished app on time with all
            planned features.
        \item Created a JavaFX app that uploads audio using Apache
            HttpClient to a REST API for transcription.
        \item Persisted transcripts using Redis streams, which allowed for
            efficient appending and distribution.
    \end{project}

    \item\begin{project}{\textbf{Blog} $|$ \emph{Rust, Docker, Github Actions, Grafana, nginx, HTML and CSS}}{}
        % FIXME Boring "used x" bullets
        \item Continuously deployed new articles to a VPS using Github
            Actions.
        \item Monitored metrics and logs using Grafana and Prometheus.
        \item Created a reverse proxy with nginx to host the blog and
            Grafana on the same domain.
        \item Used HTML, CSS, and Handlebars.js templates for the frontend.
    \end{project}

    \item\begin{project}{\textbf{Cryptograms} $|$ \emph{Rust, GraphQL, Docker, PostgreSQL}}{}
        % FIXME Boring "used x" bullets
        \item Cryptograms is a GraphQL API that generates encrypted puzzles to solve.
        \item Implemented a custom test harness for setup/teardown
            functionality that was missing in the built-in harness.
        \item Maintained test coverage greater than 90\% using Github
            Actions, with both unit and integration tests.
        \item Tracked active cryptograms with PostgreSQL.
    \end{project}
\end{entries}
\end{document}
